\chapter{Ολοκληρωμένο σύστημα κωδικοποιητή \& αποκωδικοποιητή}

\par Έχοντας ολοκληρώσει όλα τα επιμέρου συστήματα που αποτελούν ένα κωδικοποιητή 
αποκωδικοποιητη, τώρα πρέπει να τα συνδέσουμε όλα μεταξύ τους, ώστε να τον συνθέσουμε. 
Αρχικά παρουσιάζουμε τη διαδικασία που ακολουθήθηκε για την δημιουργία 
του κωδικοποιητή.
\section{Ολοκληρωμένος Κωδικοποιητής}
\par Παρακάτω παρουσιάζονται όλα τα βήματα που χρησιμοποιήσαμε για την σύνθεση ενός 
κωδικοποιητή και στη συνέχεια τα θα αναλύσουμε το κάθε ένα από αυτά.
\begin{enumerate}
\item Υποδειγματοληψία του αρχικού σήματος.
\item Χωρισμός του σήματος σε παράθυρα καθορισμένου μήκους.
\item Υπολογισμός των παραμέτρων του γραμμικού προβλέπτη.
\item Χρήση του συστήματος adpcm
\item Υπολογισμός μετασχηματισμού Huffman
\item Εγγραφή των παραμέτρων σε διαδυκή μορφή σε αρχείο
\end{enumerate}

\subsection{Υποδειγματοληψία}
\par Το πρώτο πράγμα που θα κάνουμε είναι να μετατρέψουμε το σήμα που έχουμε 
σε δειγματοληψία χαμηλότερης συχνότητας. Αυτό θα έχει σαν αποτέλεσμα να χρησιμοποιήσουμε
λιγότερα δείγματα για την κωδικοποίηση, καθώς η υποδειγματοληψία μειώνει το μήκος 
του αρχικού σήματος, όπως αναφέραμε και στην αρχή. Για την υποδειγματοληψία θα χρησιμοποιηθεί 
η συνάρτηση changefs.

\subsection{Δημιουργία παραθύρων}
\par Επειδή το δείγμα μας είναι πολύ μεγάλο είναι αναγκαίο να το χωρίσουμε 
σε μικρότερα ανεξάρτητα παράθυρα και να τα κωδικοποιήσουμε ανεξάρτητα το ένα από 
το άλλο. Στη συνέχεια για καθε δείγμα καλούμε τη συνάρτηση 
\begin{lstlisting}[style=MyMatlab]
 function [b, newstate] = encoder(x, state)
\end{lstlisting}
όπου x είναι το εκάστοτε παράθυρο και b η κωδικοποίηση του. 

\subsection{Υπολογισμός παραμέτρων γραμμικού προβλέπτη}
\par Μέσα στη συνάρτηση encoder θέλουμε να δημιουργήσουμε ένα γραμμικό προβλέπτη για 
κάθε παράθυρο. Αυτό το κάνουμε με τη χρήση της συνάρτησης lpcoeffs όπως αναφέραμε και 
παραπάνω.

\subsection{Χρήση Adpcm}
\par Στο σημείο αυτό θέλουμε να χρησιμοποιήσουμε την μέθοδο adpcm ώστε να μπορέσουμε 
να έχουμε τη δυνατότερη μικρή κωδικοποίηση. Για την χρήση όμως του adpcm είναι 
απαραίτητη η ύπαρξη ενός κβαντιστή. Άρα σε πρώτη φάση δημιουργούμε έναν ομοιόμορφο 
κβαντιστή γιατί χρειαζόμαστε τις στάθμες απόφασης και κβαντισμού. Στη συνέχεια 
χρησιμοποιούμε την συνάρτηση adpcm και υπολογίζουμε τα σύμβολα τόσο για το σήμα, 


